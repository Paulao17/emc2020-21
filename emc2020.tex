\documentclass[11pt,a4paper]{report}
\usepackage[utf8]{inputenc}
\usepackage[french]{babel}
\author{Benjamin Pajusco et Paul ALNET}
\title{Systèmes électoraux: enjeux et  emplois au sein des démocraties}
\date{2020}

% Pour citer sans que cela n'aparaisse on utilise \nocite{bibid}
% On pourra éventuellement les changer automatiquement en \cite

\begin{document}
	
\maketitle

\textbf{Quelles sont les différentes approches pour donner la voix au peuple, leurs évolutions et quelles sont les conséquences de leurs implémentations?}

\section*{Introduction}
 L'avenant des gouvernements, la transition progressive vers des états plus démocratiques, le pouvoir théoriquement donné au au peuple, nécessite l'apport d'une méthode pour permettre à chaque citoyen de faire entendre sa voix, son opinion. 
 
 \nocite{wiki:demo}

\tableofcontents

\chapter{Les systèmes électoraux dans l'Histoire}
\subsection{La vraie chose}


\section{Leurs emplois dans les démocraties modernes}

\newpage

\bibliography{bibliography} 
\bibliographystyle{ieeetr}

\end{document}
